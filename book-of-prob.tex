\documentclass[]{book}
\usepackage{lmodern}
\usepackage{amssymb,amsmath}
\usepackage{ifxetex,ifluatex}
\usepackage{fixltx2e} % provides \textsubscript
\ifnum 0\ifxetex 1\fi\ifluatex 1\fi=0 % if pdftex
  \usepackage[T1]{fontenc}
  \usepackage[utf8]{inputenc}
\else % if luatex or xelatex
  \ifxetex
    \usepackage{mathspec}
  \else
    \usepackage{fontspec}
  \fi
  \defaultfontfeatures{Ligatures=TeX,Scale=MatchLowercase}
\fi
% use upquote if available, for straight quotes in verbatim environments
\IfFileExists{upquote.sty}{\usepackage{upquote}}{}
% use microtype if available
\IfFileExists{microtype.sty}{%
\usepackage{microtype}
\UseMicrotypeSet[protrusion]{basicmath} % disable protrusion for tt fonts
}{}
\usepackage[margin=1in]{geometry}
\usepackage{hyperref}
\hypersetup{unicode=true,
            pdftitle={Book of Prob},
            pdfauthor={Berk Orbay},
            pdfborder={0 0 0},
            breaklinks=true}
\urlstyle{same}  % don't use monospace font for urls
\usepackage{natbib}
\bibliographystyle{apalike}
\usepackage{longtable,booktabs}
\usepackage{graphicx,grffile}
\makeatletter
\def\maxwidth{\ifdim\Gin@nat@width>\linewidth\linewidth\else\Gin@nat@width\fi}
\def\maxheight{\ifdim\Gin@nat@height>\textheight\textheight\else\Gin@nat@height\fi}
\makeatother
% Scale images if necessary, so that they will not overflow the page
% margins by default, and it is still possible to overwrite the defaults
% using explicit options in \includegraphics[width, height, ...]{}
\setkeys{Gin}{width=\maxwidth,height=\maxheight,keepaspectratio}
\IfFileExists{parskip.sty}{%
\usepackage{parskip}
}{% else
\setlength{\parindent}{0pt}
\setlength{\parskip}{6pt plus 2pt minus 1pt}
}
\setlength{\emergencystretch}{3em}  % prevent overfull lines
\providecommand{\tightlist}{%
  \setlength{\itemsep}{0pt}\setlength{\parskip}{0pt}}
\setcounter{secnumdepth}{5}
% Redefines (sub)paragraphs to behave more like sections
\ifx\paragraph\undefined\else
\let\oldparagraph\paragraph
\renewcommand{\paragraph}[1]{\oldparagraph{#1}\mbox{}}
\fi
\ifx\subparagraph\undefined\else
\let\oldsubparagraph\subparagraph
\renewcommand{\subparagraph}[1]{\oldsubparagraph{#1}\mbox{}}
\fi

%%% Use protect on footnotes to avoid problems with footnotes in titles
\let\rmarkdownfootnote\footnote%
\def\footnote{\protect\rmarkdownfootnote}

%%% Change title format to be more compact
\usepackage{titling}

% Create subtitle command for use in maketitle
\newcommand{\subtitle}[1]{
  \posttitle{
    \begin{center}\large#1\end{center}
    }
}

\setlength{\droptitle}{-2em}
  \title{Book of Prob}
  \pretitle{\vspace{\droptitle}\centering\huge}
  \posttitle{\par}
  \author{Berk Orbay}
  \preauthor{\centering\large\emph}
  \postauthor{\par}
  \predate{\centering\large\emph}
  \postdate{\par}
  \date{2018-03-24}

\usepackage{booktabs}
\usepackage{amsthm}
\makeatletter
\def\thm@space@setup{%
  \thm@preskip=8pt plus 2pt minus 4pt
  \thm@postskip=\thm@preskip
}
\makeatother

\usepackage{amsthm}
\newtheorem{theorem}{Theorem}[chapter]
\newtheorem{lemma}{Lemma}[chapter]
\theoremstyle{definition}
\newtheorem{definition}{Definition}[chapter]
\newtheorem{corollary}{Corollary}[chapter]
\newtheorem{proposition}{Proposition}[chapter]
\theoremstyle{definition}
\newtheorem{example}{Example}[chapter]
\theoremstyle{definition}
\newtheorem{exercise}{Exercise}[chapter]
\theoremstyle{remark}
\newtheorem*{remark}{Remark}
\newtheorem*{solution}{Solution}
\begin{document}
\maketitle

{
\setcounter{tocdepth}{1}
\tableofcontents
}
\hypertarget{introduction}{%
\chapter*{Introduction}\label{introduction}}
\addcontentsline{toc}{chapter}{Introduction}

\hypertarget{intro}{%
\chapter{Initial Concepts of Probability}\label{intro}}

\begin{itemize}
\item
  \textbf{Probability} is the quantification of event uncertainty. For
  instance, probability of getting (H)eads in a coin toss is \(1/2\).
  Deterministic models will give the same results given the same inputs
  (e.g.~2 times 2 is 4), but probabilistic models might yield different
  outcomes.
\item
  An \textbf{experiment} is a process that generates data. For instance,
  tossing a coin is an experiment. \textbf{Outcome} is the realization
  of an experiment. Possible outcomes for a coin toss is Heads and
  Tails.
\item
  \textbf{Sample space} (\(\mathbb{S}\)) is the collection of all the
  possible outcomes of an experiment. Sample space of the coin toss is
  \(\mathbb{S} = \{H,T\}\). Sample space of two coin tosses experiment
  is \(\mathbb{S} = \{HH,HT,TH,TT\}\). Sample space can be discrete
  (i.e.~coin tosses) as well as continuous (i.e.~All real numbers
  between 1 and 3.
  \(\mathbb{S} = \{x | 1 \le x \le 3, x \in \mathbb{R}\}\)) \emph{(Side
  note: Sample space is not always well defined.)}
\item
  An \textbf{event} is a subset of sample space. While outcome
  represents a realization, event is an information. Probability of an
  event \(P(A)\), say getting two Heads in two coin tosses is
  \(P(A) = 1/4\).
\item
  A \textbf{random variable} represents an event is dependent on a
  probabilistic process. On the other hand, a \textbf{deterministic
  variable} is either a constant or a decision variable. For instance,
  value of the dollar tomorrow can be considered a random variable but
  the amount I will invest is a decision variable (subject to no
  probabilistic process) and spot (current) price of the dollar is a
  constant.
\end{itemize}

\hypertarget{set-operations}{%
\section{Set Operations}\label{set-operations}}

\begin{itemize}
\item
  \textbf{Complement} of an event (\(A^\prime\)) with respect to the
  sample space represents all elements of the sample space that are not
  included by the event (A). For instance, complement of event
  \(A=\{HH\}\) is \(A^\prime=\{HT,TH,TT\}\)
\item
  \textbf{Union} of two events \(A\) and \(B\) (\(A \cup B\)) is a set
  of events which contains all elements of the respective events. For
  example, say \(A\) is the set that contains events which double Heads
  occur (\(A = \{HH,HT,TH\}\)) and \(B\) is the set which Tails occur at
  least once (\(B = \{TT,HT,TH\}\)). The union is
  \(A \cup B = \{HH,TH,HT,TT\}\).
\item
  \textbf{Intersection} of two events \(A\) and \(B\) (\(A \cap B\))
  contains the common elements of the events. For example, say \(A\) is
  the set that contains events which Heads occur at least once
  (\(A = \{HH,HT,TH\}\)) and \(B\) is the set which Tails occur at least
  once (\(B = \{TT,HT,TH\}\)). The intersection is
  \(A \cap B = \{TH,HT\}\).
\item
  \textbf{Mutually exclusive} or disjoint events mean that two events
  have empty intersection (\(A \cap B = \emptyset\)) and their union
  (\(A \cup B\)) contains the same amount of elements as the sum of
  their respective number of elements. Also \(P(A \cap B) = 0\) and
  \(P(A \cup B) = P(A) + P(B)\). For example getting double Heads
  (\(HH\)) and double Tails (\(TT\)) are mutually exclusive events.
\end{itemize}

\hypertarget{axioms-of-probability}{%
\section{Axioms of Probability}\label{axioms-of-probability}}

\begin{enumerate}
\def\labelenumi{\arabic{enumi}.}
\tightlist
\item
  Any event \(A\) belonging to the sample space \(A \in \mathbb{S}\)
  should have nonnegative probability (\(P(A) \ge 0\)).
\item
  Probability of the sample space is one (\(P(\mathbb{S}) = 1\)).
\item
  Any disjoint events
  (\(A_i \cap A_j = \emptyset \ \forall_{i,j \in 1 \dots n}\)) satisfies
  \(P(A_1 \cup A_2 \cup \dots \cup A_n) = P(A_1) + P(A_2) + \dots + P(A_n)\).
\end{enumerate}

\hypertarget{other-set-and-probability-rules}{%
\section{Other Set and Probability
Rules}\label{other-set-and-probability-rules}}

\begin{itemize}
\item
  \((A^\prime)^\prime = A\)
\item
  \(S^\prime = \emptyset\)
\item
  \(\emptyset^\prime = S\)
\item
  \((A \cap B)^\prime = A^\prime \cup B^\prime\)
\item
  \((A \cup B)^\prime = A^\prime \cap B^\prime\)
\item
  \((A \cup B) \cap C = (A \cap C) \cup (B \cap C)\)
\item
  \((A \cap B) \cup C = (A \cup C) \cap (B \cup C)\)
\item
  \((A \cup B) \cup C = A \cup (C \cup B)\)
\item
  \((A \cap B) \cap C = A \cap (C \cap B)\)
\item
  \(A \cup A^\prime = \mathbb{S}\) and \(A \cap A^\prime = \emptyset\)
  so \(P(A) = 1 - P(A^\prime)\). This is especially useful for many
  problems. For example the probability of getting at least one Heads in
  a three coin tosses in a row is \(1 - P(\{TTT\}) = 7/8\), the
  complement of no Heads in a three coin tosses in a row. Otherwise, you
  should calculate the following expression.

  \[P(\{HTT\}) + P(\{THT\}) + P(\{TTH\}) + P(\{HHT\}) + P(\{HTH\}) + P(\{THH\}) + P(\{HHH\}) = 7/8 \]
\item
  If \(A \subseteq B\)~then \(P(A) \le P(B)\).
\item
  \(P(A \cup B) = P(A) + P(B) - P(A \cap B)\).
\item
  \(P(A \cup B \cup C) = P(A) + P(B) + P(C) - P(A \cap B) - P(B \cap C) - P(A \cap C) + P(A \cap B \cap C)\)
\end{itemize}

\hypertarget{counting}{%
\section{Counting}\label{counting}}

Counting rules will help us enumerate the sample space. It will include
multiplication rule, permutation and combination.

\hypertarget{multiplication-rule}{%
\subsection{Multiplication Rule}\label{multiplication-rule}}

If I have a series of independent events, say \(1\) to \(k\), and number
of possible outcomes are denoted with \(n_1\) to \(n_k\); total number
of outcomes in the sample space would be \(n_1n_2\dots n_k\).

Take a series of coin tosses in a row. If I toss a coin its sample space
consists of 2 elements such as \(\{H,T\}\). If I toss 2 coins the sample
space would be 2*2 \(\{HH,HT,TH,TT\}\). If I toss 3 coins, the sample
space would be 2*2*2 \(\{HHH,HTH,THH,TTH,HHT,HTT,THT,TTT\}\).

A poker card consists of a type and a rank. There are four types of
playing cards (clubs, diamonds, hearts and spades) and 13 ranks (A - 2
to 10 - J - Q - K). Number of cards in a deck is 4*13 = 52.

\hypertarget{permutation-rule}{%
\subsection{Permutation Rule}\label{permutation-rule}}

Permutation is the arrangement of all or a subset of items.

\begin{itemize}
\item
  Given a set of items, say \(A = {a,b,c}\) in how many different ways I
  can order the elements? Answer is n!. In our case it is,
  \(3! = 3.2.1 = 6\).

  \[A = \{a,b,c\},\{b,a,c\},\{b,c,a\},\{c,a,b\},\{c,b,a\},\{a,c,b\}\]
\item
  Suppose there are 10 (n) participants in a competition and 3 (r)
  medals (gold, silver and bronze). How many possible outcomes are
  there? Answer is
  \(n(n-1)(n-2)\dots (n-r+1) = \dfrac{n!}{(n-r)!} = \dfrac{10!}{(10-3)!} = 720\).
\item
  If there are more than one same type items in a sample, then the
  permutation becomes \(\dfrac{n!}{n_1!n_2!\dots n_k!}\) where
  \(\sum n_i = n\).

  For example enumerate the different outcomes of four coin tosses which
  result in 2 heads and 2 tails. Answer is \(\dfrac{4!}{2!2!} = 6\)

  \[A = \{HHTT,HTTH,HTHT,THTH,THHT,TTHH\}\]
\end{itemize}

\hypertarget{combination-rule}{%
\subsection{Combination Rule}\label{combination-rule}}

Suppose we want to select \(r\) items from \(n\) items and the order
does not matter. So the number of different outcomes can be found using
\(\binom{n}{r} = \dfrac{n!}{(n-r)!r!}\).

Out of 10 students how many different groups of 2 students can we
generate? Answer \(\dfrac{10!}{8!2!} = 45\)

\hypertarget{literature}{%
\chapter{Literature}\label{literature}}

Here is a review of existing methods.

\hypertarget{methods}{%
\chapter{Methods}\label{methods}}

We describe our methods in this chapter.

\hypertarget{applications}{%
\chapter{Applications}\label{applications}}

Some \emph{significant} applications are demonstrated in this chapter.

\hypertarget{example-one}{%
\section{Example one}\label{example-one}}

\hypertarget{example-two}{%
\section{Example two}\label{example-two}}

\hypertarget{final-words}{%
\chapter{Final Words}\label{final-words}}

We have finished a nice book.

\bibliography{book.bib,packages.bib}


\end{document}
